\section{Introduction}
\label{introduction}
%\begin{introduction}
%\boldmath
A wireless ad-hoc network is a collection of several mobile nodes that dynamically form a network for communication without any pre-existing infrastructure or any centralized control. This network is popularly used in emergency search-and-rescue operations, decision making in the battlefield etc. where network needs to be deployed immediately but the base stations or fixed network or infrastructures are not available. The mobile nodes often cannot communicate with each other directly because of several causes such as limitation of power, improper utilization of channel and many more. So, they rely on several intermediate nodes (known as relays) to exchange data with one another over the network and thus the wireless multi-hop network is formed.
 %The mobility of the nodes causes continuous change in network topology which is the greatest concerns in wireless multi-hop network. 
 The nodes in these networks frequently need to broadcast messages for route discovery, periodic data dissemination, erasing an invalid route, locating a node,  or even for sending alarm signals in the entire network.
 
An effortless approach to perform broadcast is \emph{blind flooding}. Every node forwards a broadcast message exactly once in blind flooding. Though blind flooding ensures full coverage at high mobility but unfortunately, it results in redundant traffic, contention, and collision and finally leads to \textit{broadcast storm problem}\cite{tseng2002broadcast}. One solution to overcome these problems is to compute a virtual backbone based on the physical topology, and run any existing routing protocol over the virtual backbone\cite{butenko2004new}. 

The Connected Dominating Set (CDS) is a widely used approach in this context. A CDS is generally constructed by at first modeling the entire network as a graph $G$ where all nodes form a node set $V$ and the communication links between nodes form an edge set $E$. Then, a CDS of a graph $G$ contains a subset $D$ of the node set $V$ where any node in $D$ can reach any other node in $D$ by a path that stays entirely within $D$. That is, $D$ induces a connected sub-graph of $G$ and every node in $G$ either belongs to $D$ or is adjacent to a node in $D$.  Only the nodes belonging to CDS  participate in forwarding to convey the message in the entire network.

%hence they are known as the forwarding nodes of the network.
 A notable number of centralized and distributed algorithms ~\cite{yang2017routing, butenko2004new, luo2017new} have been devised to create CDS to reduce the number of packet forwarding. Although CDS based algorithms have been proposed to reduce redundancy, none of these works aim at minimizing contention. Contention means \textit{competition for resources}. The term is used especially in networks to describe the situation where two or more nodes attempt to transmit a message across the same medium at the same time. In a mobile ad-hoc network, after broadcasting a message by a mobile host, if more than one neighbors which are within their transmission range want to rebroadcast it, these transmissions may face serious contention with each other.
 
 The main contribution in this paper is to construct a connected dominating set in such a way so that the contention among the nodes in CDS is minimized. A couple of new heuristics have been proposed to construct the contention aware connected dominating set (CACDS) in this work. 
 
 A centralized algorithm has been deduced first to construct CDS using the global topology information of a wireless network. Since it is difficult to gather the entire topology information, a distributed version has also been presented where a node selects a subset of nodes from its immediate neighbors as forwarding nodes based on 2-hop neighborhood information. A demonstration of the efficiency of the distributed algorithm using the centralized algorithm as a benchmark is also provided in this work. Finally, a comprehensive simulation to analyze the behavior of the proposed algorithms and compare their performances with other state-of-the-art algorithms in terms of number of forwarding nodes and amount of contention has been presented. In centralized environment, with the increase of 0-5\% forwarding nodes, new heuristic generates almost 90-100\% contention free CDS and it increases the number of forwarding nodes by 1-5\% to mitigate almost 4-19\% contention in CDS in a distributed environment.

The rest of the paper is organized as follows. In Section 2, the basic idea of this work has been discussed. In Section 3, we present the state-of-the-art research works. Section 4 provides the important terminologies and considered state-of-art algorithms. Section 5 deals with the methodology of the new proposed algorithms with suitable examples. Section 6 demonstrates the simulation and performance evaluation and finally Section 7 concludes the paper highlighting the contribution, limitations  and possible future works.  
%\end{introduction}


