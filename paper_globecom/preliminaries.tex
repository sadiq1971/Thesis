\section{Preliminaries}
\label{Preliminaries}

We use a simple graph $G(V,E)$ to represent an ad-hoc network, where $V$ represents the set of wireless mobile nodes and $E$ represents the set of edges. An edge $(u,v)$ indicates that both hosts $u$ and $v$ are within their transmission range. $N(u)$ is defined as a set of adjacent nodes of node $u$ and $N(N(u))$ is the set of nodes that is at most 2-hop away from node $u$. $F\textsubscript{u}$ represents the list of 1-hop neighbors of $u$ that are selected for forwarding by node $u$. $B\textsubscript{u}$ is the set of 1-hop neighbors of node $u$ that are eligible to be included in the forwarding list $F\textsubscript{u}$. $U\textsubscript{u}$ is the set of nodes that need to be covered by using nodes from $B\textsubscript{u}$ while $u$ creates its forwarding list $F\textsubscript{u}$ . 

MCDS \cite{guha1998approximation} and Dominant Pruning \cite{lim2001flooding} has been used as state-of-art algorithms to evaluate the new proposed approaches. 

\textbf{(i) MCDS Construction Algorithm:} At the start of the algorithm, all nodes in the network are colored white. The node with maximum cardinality is then selected and colored black. All the one-hop neighbors of that node are colored gray. A gray node having maximum number of white neighbors is then selected and colored black. The selection process recursively runs until no white node exits. The set with all the black nodes are the resultant nodes that makes MCDS. In figure \ref{aa}(a), $$MCDS = \{A,B,C\}$$.
 
\textbf{(ii) Dominant Pruning:} The forwarding list creation process of dominant pruning algorithm is as follows:- suppose, a node $v$ receives a packet from node $u$. The sender node $u$ also sends a \textit{forwarding list ($F\textsubscript{u}$)}  with the packet header. If  $v \in F\textsubscript{u}$, then the node $v$ will rebroadcast and will create its own forward list ($F\textsubscript{v}$). The node then start constructing $U\textsubscript{v}$ which is all uncovered two-hop neighbors of $v$. The set $B\textsubscript{v}$ represents those neighbors of $v$ which are possible candidates for inclusion in $F\textsubscript{v}$. Then, in each iteration, $v$ selects a neighbor $w$ $\in$ $B\textsubscript{v}$, such that $w \notin F\textsubscript{v}$ and the list of neighbors of $w$ covers the maximum number of nodes in $U\textsubscript{v}$ , i.e $|N(w)\cap U\textsubscript{v}|$ is maximized. Next $v$ includes $w$ in  $F\textsubscript{v}$ and sets $U\textsubscript{v} =U\textsubscript{v} - N(w)$ . The iterations continue for as long as $U\textsubscript{v}$ becomes empty or no more progress can be accomplished. 
 